\chapter*{Ringraziamenti}

Alla fine di questo elaborato ho il piacere di ringraziare tutti quelli che hanno contribuito in questi anni di carriera universitaria.\\

In primis, ringrazio la mia famiglia: mia madre, mio padre, mia sorella e i miei nonni; voi mi avete incoraggiato e avete creduto in me,
 anche quando io stesso avevo dei dubbi. In ogni momento difficile, in cui qualsiasi altra persona mi avrebbe abbandonato, voi ci siete stati,
  lì pronti per me. Quando dopo le superiori volevo mollare e ho fatto sbagli uno dietro l’altro, quando durante l’università volevo arrendermi, 
  voi mi avete sempre sostenuto, in modo discreto, con quella classica “spinta” che non ti accorgi di avere ma che ti fa andare avanti.\\ 
  Mi sono ripromesso che non farò più errori e che andrò avanti fino alla fine, per voi. Grazie.\\

Un ulteriore ringraziamento va ai miei cugini che mi hanno sempre appoggiato, soprattutto in questi ultimi anni: le chiacchierate pomeridiane
 parlando dei nostri esami e le giornate di studio intenso insieme rimarranno sempre scolpite nella mente.\\

 \begin{CJK}{UTF8}{min}
特別なありがとうはしーちゃんに。知り合ってからそんなに時間は経ってないし、一緒に過ごした時間も多くなかったけど、Campus Hubや図書館、うちで一緒に勉強した午後はずっと心に残ると思う。\\
 君と出会ったこと、そして一緒の未来を想像することが、前に進む理由をくれた。本当にありがとう。
 \end{CJK}
 \\

Ai miei amici di Catania, che ci sono sempre stati nello sconforto e nella gioia, che mi hanno sempre appoggiato e incoraggiato, grazie.\\
 Negli anni gli amici si incontrano e si perdono, a noi è andata così. \\
 Ma alla fine l’importante è chi rimane, e voi siete rimasti. Grazie per le belle serate, le francescane bevute insieme dopo ogni esame e le risate.\\

Ai miei amici di Malta, che mi hanno fatto passare un anno indimenticabile: siamo stati una vera famiglia, in un’isola che ci ha lasciato ricordi indelebili.\\
 Ci siamo divertiti e abbiamo studiato pomeriggi interi insieme, ci siamo aiutati a vicenda e siete stati una seconda famiglia quando ero lontano da casa, grazie.

Ulteriori ringraziamenti sono rivolti a quei docenti che in questi anni hanno saputo trasmettere con passione, dedizione e preparazione
 l’amore che essi stessi nutrono per le loro materie. Grazie al loro impegno e alla loro capacità di coinvolgere hanno creato ambienti stimolanti,
  capaci di lasciare a tutti gli studenti, me compreso, amore e passione per gli insegnamenti svolti.\\

Infine, un ringraziamento va al mio relatore, che mi ha sostenuto in ogni fase del lavoro, lasciandomi la libertà di seguire le mie idee e 
incoraggiandomi a svilupparle senza paura. La sua guida non è mai stata un vincolo, ma un invito a crescere, a migliorarmi, a credere nelle mie capacità.\\
 E quando, alla fine, mi ha proposto la pubblicazione e la partecipazione a una conferenza, che ha dato ancora più valore a questo percorso.\\
