\chapter{Conclusioni}
Il percorso intrapreso ha permesso di verificare che la scelta di un canale acustico per collegare nodi master--slave in ambienti sotterranei 
è realistica quando le onde elettromagnetiche risultano inaffidabili.\\
La progettazione è partita dall'analisi delle condizioni di propagazione 
e dei vincoli strutturali descritti nei capitoli iniziali \autoref{chap:introduzione} , \autoref{chap:stato_arte}, 
traducendosi in un protocollo capace di sfruttare coppie di frequenze dedicate e schemi
 di conferma rapidi gestiti da handles ad ACK come descritto nella \autoref{sec:livello_link}.\\
Le scelte progettuali sono state costantemente verificate con prove sperimentali diverse, sono state effettuate prove di emissione e verifica della 
stabilità del suono al fine di evitare fruscii come evidenziato nella \autoref{sec:uscita_livello_fisico}, 
poi si è passati alla verifica del riconoscimento delle frequenze, che ha richiesto 4 mesi di lavoro, con l'ausilio di un trasduttore 
piezoelettrico collegato ad un amplificatore Classe-D da \SI{230}{\volt} a sua volta collegato all'uscita AUX di un computer;
 si è poi proceduto alla verifica del protocollo completo in ambienti controllati e infine in un corridoio sotterraneo di 
15 metri con curve a 90 gradi con l'ausilio di un fonometro.\\

La parte hardware e software si è evoluta insieme, con il microcontrollore ESP32 che ha offerto abbastanza margine per campionare a 
\SI{48}{\kilo\hertz}, eseguire l'FFT a 512 campioni e mantenere in tempo reale il riconoscimento delle frequenze utili, più la derivante elaborazione 
del pacchetto dai.\\
L'implementaziomne hardware discussa nel \autoref{chap:implementazione_hardware} ha permesso di realizzare un nodo economico, con un costo
 totale inferiore a 20 euro per unità, che integra un microfono MEMS e un DAC I2S con amplificatore di potenza in Classe-D,
 il tutto alimentato a \SI{3.3}{\volt} con un consumo di picco inferiore a \SI{200}{\milli\ampere} durante la trasmissione, risultando 
 perfettamente in linea con quanto previsto nei requisiti iniziali (\autoref{sec:contributi_attesi}).\\


 \begin{figure}[H]
    \centering
    \begin{adjustbox}{width=0.9\linewidth}
    \begin{tikzpicture}[
        node distance=3cm,
        every node/.style={
            rounded corners,
            draw=black,
            align=center,
            minimum width=3.2cm,
            minimum height=1cm,
            font=\small
        }
    ]
        % Nodi
        \node[fill=blue!15] (analisi) {Analisi del canale\\e requisiti acustici};
        \node[fill=green!15, right=of analisi] (protocollo) {Design protocollo\\frequenze e ACK};
        \node[fill=orange!20, below=of protocollo] (hw) {Implementazione\\hardware ESP32 + MEMS};
        \node[fill=yellow!25, right=of hw] (sw) {Firmware FFT\\e compressione run-length};
        \node[fill=red!15, below=of hw] (test) {Sperimentazione\\bitrate e attenuazioni};
        \node[fill=gray!20, left=of hw] (applicazioni) {Applicazioni\\monitoraggio post-disastro};

        % Frecce
        \draw[->, thick] (analisi) -- (protocollo);
        \draw[->, thick] (protocollo) -- (hw);
        \draw[->, thick] (hw) -- (test);
        \draw[->, thick] (test) -- (applicazioni);
        \draw[->, thick, dashed] (applicazioni) -- (analisi);
        \draw[->, thick] (protocollo) -- (sw);
        \draw[->, thick] (sw) -- (test);
    \end{tikzpicture}
\end{adjustbox}
    \caption{Flusso progettuale e sperimentale della tesi.}
    \label{fig:conclusioni_flusso}
\end{figure}


Le prove in ambiente controllato e nel corridoio sotterraneo hanno confermato che, con un'emissione iniziale di circa \SI{70}{\decibel_{SPL}}, 
la perdita lungo quindici metri resta nell'ordine di \SI{15}{\decibel} e la differenza tra percorso rettilineo e curva a novanta gradi si traduce in 
un aumento della packet loss dall'1\% al 18\% dopo i \SI{3}{\meter} dall'imbocco laterale come è mostrato nella \autoref{fig:profilo_db}.\\
 Il protocollo ha mantenuto un bitrate utile medio di \SI{14.4}{\bit\per\second} con una percentuale di ritrasmissioni sotto l'1\%,
  valori che per un sistema ancora in stato embionale, creato con pochissimi fondi e materiale di recupero, alimentato a \SI{3.3}{\volt} risultano incoraggianti.\\
   Le registrazioni di sequenze REQ/SET/OK e le trasmissioni telemetriche hanno mostrato come l'auto-taratura 
   della soglia spettrale riesca a seguire i cambiamenti del rumore, evitando falsi positivi anche quando l'ambiente risultava particolarmente rumoroso,
   con voci di sottofondo o rumore di dispositivi elettronici accesi.\\
   Una dimostrazione pratica del funzionamento del protocollo è disponibile nel video presente nell' \autoref{appendice:repository}.\\


   Il sistema ha ancora alcuni limiti, soprattutto legati alla velocità di trasmissione
    e al fatto che il segnale si indebolisce dopo curve strette, dove l’attenuazione diventa più marcata come osservato nel \autoref{chap:sperimentazione_propagazione}.\\
     Nonostante questo, il progetto non è affatto chiuso: anzi, ci sono margini di miglioramento ben definiti citati nel \autoref{chap:sviluppi_futuri},
     che potrebbero portare ad incrementi significativi delle prestazioni, in entrambi gli aspetti sopra.

   \paragraph{In conclusione} 
   Il lavoro ha dimostrato che un protocollo \textbf{acustico low cost} può funzionare in contesti critici, raggiungendo
    l’obiettivo principale, superando in alcuni contesti pure certi standard di industria osservati nel \autoref{chap:stato_arte}.\\
     Nonostante i limiti attuali, i risultati ottenuti rappresentano una base per sviluppare versioni più robuste, rapide ed adatte ad ambienti
     diversi da quelli trattati in questa tesi, questa era volta più ad aprire una strada verso queste applicazioni che a fornire una soluzione definitiva.\\
