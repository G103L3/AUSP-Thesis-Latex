\chapter{Sviluppi futuri}
\label{chap:sviluppi_futuri}

Il progetto trattato potrà trovare nuovi sbocchi partendo dalle conclusioni raggiunte in \autoref{chap:introduzione},
 dove si è visto come il suono riesca a infilarsi nei cunicoli in cui le onde radio fanno fatica a passare.\\ 
Pensando a miniere e gallerie, ma anche a scenari di emergenza, lo stesso impianto hardware e software può diventare un supporto
 per squadre di sicurezza che devono tenere sotto controllo vibrazioni anomale, presenza di gas o segnali vitali di operatori isolati in caso di disastri.  \\
La rete acustica può guidare piccoli robot mobili controllati da remoto, lasciando ai nodi fissi il compito di fare da ripetitori tra un corridoio
 e l'altro e di aggiornare la sala operativa sulla posizione del robot, anche quando il tracciato cambia dopo un crollo.

\section{Evoluzione degli scenari applicativi}
\label{sec:scenari_applicativi}

Gli sviluppi immaginati per gli ambienti minerari sono il primo esempio concreto: schede come quella descritta in questa tesi alimentate a batteria e installate 
lungo le gallerie possono offrire un canale acustico di comunicazione per raggiungere nodi finali addibiti alla raccolta di dati ambientali (Sensori PIR o Sensori Radar Doppler)
 o al controllo di robot mobili. \\
 Il paragrafo \autoref{sec:velocita} torna su questo punto perché la velocità diventa cruciale nelle operazioni di ricerca di personale disperso;
  in parallelo, il nodo master può comandare una squadra di robot grazie al protocollo master--slave già enunciato nel \autoref{chap:design_protocollo},
   chiedendo loro di esplorare diramazioni pericolose e di tornare indietro in caso di crolli improvvisi. \\

L'idea si espande anche ad ambienti marini, adattando il l'hardware e dotandolo di trasduttori infrasonici ad alta sensibilità in grado 
gestire frequenze compresse tra 0 e \SI{20}{\hertz}, come per la versione "terrestre" ispirati alla natura, in particolare ai pappagalli, la versione sottomarina
potrebbe sfruttare le onde ELF (Extremely Low Frequency, 3--30 Hz) per comunicare a grandi profondità, ispirandosi alle balenottere azzurre che 
usano queste frequenze per coprire distanze oceaniche \citep[sec.~2]{Dziak2017}, questo permetterebbe la comunicazione con sommergibili o con sensori di profondità. \\

 Studi specifici sulla propagazione in banda ELF confermano che segnali così lenti, se filtrati correttamente,
  riescono a raggiungere sommergibili che stazionano a grande profondità senza dover riemergere. \\

  In maniera quasi speculare, in una situazione estrema, il protocollo potrebbe tornare utile come backup terrestre: nel caso di una tempesta solare capace di
   mandare fuori gioco i ponti radio, una dorsale acustica basata su condensatori (microfoni digitali a condensatori) e attuatori elettrodinamici, come 
    quelli usati nei prototipi,
    potrebbe offrire un canale di comunicazione lento ma affidabile.\\

\section{Revisione del formato di pacchetto}
\label{sec:pacchetti_future}


Nell'ipotesi di questi sviluppi futuri, allora si rende necessaria una modifica del protocollo descritto in particolare nella \autoref{sec:livello_link}.\\ 
L'assetto attuale, che codifica l'intestazione come stringa \\ 
\textbf{ID:id\_destinatario{OPERAZIONE}k{id\_mittente}}, 
ha delle criticità che rendono il protocollo più lento di quello che potrebbe diventare.\\
 La riscrittura prevede di riservare un solo carattere ASCII di controllo, per esempio il simbolo 
 "\textbf{?}", così da separare i campi del pacchetto senza l'utilizzo di parentesi graffe o scritte complesse come "ID:".
  In questo modo il ricevitore non deve più cercare pattern letterali 
 ma può fermarsi alla prima occorrenza del delimitatore, impacchettando ogni blocco in una specifica categoria.\\
 La scelta del carattere però deve essere ben ponderata, questo deve essere un carattere \textbf{con un alto tasso di compressione} utilizzando l'algoritmo 
 di compressione run-length descritto nel \autoref{chap:implementazione_software}, in modo da rappresentare un separatore con il minor numero di bit possibile.\\
 Le modifiche sopra verebbero implementate nel "frame" attuale, mentre un ulteriore modifica al di fuori di quest'ultimo sarebbe la compressione \textbf{Huffman}
    del pacchetto binario così da ridurre ulteriormente la dimensione del messaggio.\\
    La compressione di Huffman è un algoritmo di compressione senza perdita che assegna codici di lunghezza variabile ai simboli in base alla loro frequenza di occorrenza. \\
     I simboli più frequenti ricevono codici più corti, mentre quelli meno frequenti ricevono codici più lunghi.
     \newpage  
     Questo permetterebbe di assegnare ai comandi più frequenti quali: delimitatori, ID di rete ed operazioni più frequenti o più importanti per l'azienda commitrice
        codici più corti, riducendo la dimensione complessiva del pacchetto.\\
        Un esempio di questa evoluzione è mostrato in \autoref{fig:roadmap_pacchetto}.

        \begin{figure}[H]
            \centering
            \begin{adjustbox}{width=0.9\linewidth}
            \begin{tikzpicture}[>=Latex]
                % Definizione degli stili
                \tikzstyle{stage}=[
                    rectangle, 
                    rounded corners, 
                    draw=blue!60, 
                    very thick, 
                    minimum width=3cm, 
                    minimum height=1cm, 
                    fill=blue!10,
                    align=center
                ]
                \tikzstyle{arrow}=[->, very thick, color=blue!80]
        
                % Matrix per i blocchi
                \matrix (m) [matrix of nodes,
                             column sep=1.5cm, % distanza orizzontale tra i blocchi
                             row sep=1cm,
                             nodes={stage}] {
                    Frame attuale "ID:..." & 
                    Delimitatore "?" & 
                    Pre-codifica Huffman &
                    Blocchi binari compatti \\
                };
        
                % Frecce tra i blocchi
                \draw[arrow] (m-1-1) -- (m-1-2);
                \draw[arrow] (m-1-2) -- (m-1-3);
                \draw[arrow] (m-1-3) -- (m-1-4);
        
            \end{tikzpicture}
        \end{adjustbox}
            \caption{Roadmap di evoluzione del frame, dal formato attuale verso blocchi compatti e pronti alla compressione run-length attuale}
            \label{fig:roadmap_pacchetto}
        \end{figure}
        

La roadmap di \autoref{fig:roadmap_pacchetto} riassume gli step principali: il lavoro sul delimitatore prepara il terreno a un lessico binario
 che potrà essere precompresso con Huffman prima di affrontare la compressione a livello link.\\

\section{Sensoristica avanzata}
\label{sec:sensori_future}
Come già accennato in \autoref{sec:scenari_applicativi}, l'adozione di sensori più sofisticati permetterebbe di estendere il protocollo a scenari più complessi,
 come le comunicazioni sottomarine o il monitoraggio ambientale in miniere profonde.\\
Per la ricezione del segnale, si rende indispensabile sostituire i microfoni MEMS standard con microbarometri capacitivi.\\
 In parallelo, un convertitore sigma-delta, attualmente limitato a 16 bit, potrebbe essere cambiato con un modello a 24 bit, in grado di
 campionare anche frequenze ultrabasse, rendendo il sistema a livello hardware in grado di operare in banda ELF, quindi implementarlo per usi sottomarini.\\ 

Dal lato della trasmissione, invece, sarebbe meglio adottare amplificatori di classe-D come quello attulamente utilizzato, 
ma con una potenza di uscita più elevata, in modo da poter pilotare trasduttori elettrodinamici più grandi e in grado di spostare più aria o acqua.\\


\section{Riduzione dei tempi di segnalazione}
\label{sec:velocita}

Attualmente ogni signal code occupa \SI{152}{\milli\second}, somma dell'emissione ripetuta tre volte e della finestra di silenzio da \SI{80}{\milli\second} 
 come mostrato nella sequenza di \autoref{fig:timeline_pacchetto}. 

Utilizzando l'elettronica discussa nella sezione precedente \autoref{sec:sensori_future} potrebbe essere possibile, allora, la riduzione
del tempo di emissione per ogni signal code, presumibilmente, nel caso migliore emettendolo solo per \SI{10.7}{\milli\second} 
(tempo di campionamento discusso nella \autoref{sec:livello_fisico})
susseguito da un silenzio di \SI{10.7}{\milli\second} per un totale di  \SI{21.4}{\milli\second}. \\
 

Questo porterebbe diversi vantaggi, oltre ad un throughput migliore, che si attesterebbe intorno a 47.6 signal codes per secondo, che con una compressione 
run-length di circa 4:1, come discusso nella \autoref{sec:livello_link}, porterebbe a circa 190 bits per secondo.
